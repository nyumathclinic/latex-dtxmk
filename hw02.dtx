%<*driver>
% !TEX program = latexmk
% !TEX options = -r dtxmk.latexmkrc -synctex=1 -interaction=nonstopmode -file-line-error -pdf %DOC%
\input docstrip.tex
\askforoverwritefalse
\edef\qnspbl{\perCent\space This is a LaTeX file.  It is a text file that is compiled^^J%
\perCent\space by a program called LaTeX into a pretty PDF file.^^J%
\perCent\space If you're viewing this file in Overleaf, you'll see that PDF^^J%
\perCent\space in the window to the right.^^J%
\perCent^^J%
\perCent\space This file is for typesetting the QUESTIONS in this homework assignment.^^J%
\perCent\space You can read the source code to learn more about how specific math^^J%
\perCent\space expressions are encoded in the LaTeX language. But you will edit a copy^^J%
\perCent\space of the file '\jobname.ans.tex' instead.^^J%
\perCent}
\edef\anspbl{\perCent\space This is a LaTeX file.  It is a text file that is compiled^^J%
\perCent\space by a program called LaTeX into a pretty PDF file.^^J%
\perCent\space If you're viewing this file in Overleaf, you'll see that PDF^^J%
\perCent\space in the window to the right.^^J%
\perCent^^J%
\perCent\space This file is for typesetting YOUR ANSWERS to this homework assignment.^^J%
\perCent\space The LaTeX macro language is complicated, so we have inserted lots of^^J%
\perCent\space documenting comments into the file.  Comments start with '\perCent'^^J%
\perCent\space and continue to the end of the line.  In Overleaf's edit window, they^^J%
\perCent\space are colored green.^^J%
\perCent\space ^^J%
\perCent\space Comments prefixed with 'Student:' are relevant to students. Skip any-^^J%
\perCent\space thing else you don't understand, or ask me.^^J%
}
\edef\solpbl{\perCent\space This is a LaTeX file.  It is a text file that is compiled^^J%
\perCent\space by a program called LaTeX into a pretty PDF file.^^J%
\perCent\space If you're viewing this file in Overleaf, you'll see that PDF^^J%
\perCent\space in the window to the right.^^J%
\perCent^^J%
\perCent\space This file is for typesetting SOLUTIONS AND REMARKS for this homework^^J
\perCent\space assignment.  As with the questions file, you can read the source code^^J%
\perCent\space if you want to know how I typeset the math that appears in the PDF.%
}
\generate{
    \usepreamble\qnspbl\file{\jobname.qns.tex}{\from{\jobname.dtx}{questions}}
    \usepreamble\anspbl\file{\jobname.ans.tex}{\from{\jobname.dtx}{questions,answers}}
    \usepreamble\solpbl\file{\jobname.sol.tex}{\from{\jobname.dtx}{questions,solutions}}
}
\endbatchfile
%</driver>
%<*questions>
\documentclass{article}
%% This is some font management depending on the TeX “engine” being used.
%% Nothing to worry about.
\usepackage{ifxetex}
\ifxetex
  \usepackage{fontspec}
\else
  \usepackage[T1]{fontenc}
  \usepackage[utf8]{inputenc}
  \usepackage{lmodern}
\fi

%% Student: These lines describe some document metadata.
\title{Homework Assignment 2}
\usepackage{etoolbox}
%<answers>\makeatletter\preto{\@title}{Answers to }\makeatother
%<solutions>\makeatletter\preto{\@title}{Solutions to }\makeatother
\author{%
%<!answers>    Prof. Leingang
%<*answers>
\GenericWarning{}{LaTeX Warning: Please change ``Sally Student'' to your name}% Delete this line when you have done this
    Sally Student
%</answers>
\\  MATH-UA 120.005 Discrete Mathematics
}
\date{due September 13 at 12:00noon}


%% These lines set up the question, answer, and solution environments.
\usepackage{amsthm}
\usepackage{amssymb}
%<!answers&!solutions>\theoremstyle{definition}
%<answers|solutions>\theoremstyle{plain}
\newtheorem{question}{Question}

\newenvironment{answer}[1][Answer]
    {\begin{proof}[#1]\renewcommand\qedsymbol{$\vartriangle$}}
    {\end{proof}}
\newenvironment{solution}[1][Solution]
    {\begin{proof}[#1]\renewcommand\qedsymbol{$\blacktriangle$}}
    {\end{proof}}
% These make enumerates within questions start at the second ("(a)") level, 
% rather than the first ("1.") level.
\makeatletter
    \newcommand{\stepenumdepth}{\advance\@enumdepth\@ne}
\makeatother
\AtBeginEnvironment{question}{\stepenumdepth}
\AtBeginEnvironment{answer}{\stepenumdepth}
\AtBeginEnvironment{solution}{\stepenumdepth}

\usepackage{amsmath}
\usepackage{siunitx}

\usepackage{hyperref}
%% This is the beginning of the part of the file that describes
%% the actual text of the document.
%% That's why it says `\begin{document}' below. :-)
\begin{document}
\maketitle

%<*!answers&!solutions>
\section*{Reading and Practice}

Read sections 6--9 of the textbook.  

\begin{itemize}
\item In Section 6, make sure you understand what a counterexample is, and how to use it to disprove a statement
\item In Section 7, know the basic Boolean algebra operations and identities, and how to prove equations with and without truth tables
\item In Section 8, know how to count lists, and the multiplication principle
\item In Section 9, know how to count rearrangements using factorials.
\end{itemize}


Try these problems for practice: 
\begin{itemize}
    \item Section 6: 2, 4, 5, 7, 15
    \item Section 7: 1, 2, 10, 11, 12
    \item Section 8: 1, 3, 5, 7, 9
    \item Section 9: 1, 5, 7
\end{itemize}    
   

\section*{Assigned Problems}

%</!answers&!solutions>

\begin{question}
    Scheinerman, Exercise 6.6:
    Disprove: if $p$ is prime, then $2^p-1$ is also prime.
\end{question}
%<*answers>
%% Student: put your answer between the next two lines.
\begin{answer}
\end{answer}
%</answers>
%<*solutions>
\begin{solution}
    Let $p=11$.  Then $p$ is prime.  But $2^p-1 = 2^{11}-1 = 2047 = 23 \times 89$.
    So the statement is false.
    
    A prime number that is equal to $2^n-1$ for some $n$ is called a \href{https://en.wikipedia.org/wiki/Mersenne_prime}{\emph{Mersenne Prime}}.
\end{solution}
%</solutions>

\begin{question}
    Scheinerman, Exercise 6.8:
    An integer is called a \emph{palindrome} provided
    it reads the same forwards and backwards in its base-ten
    representation.  For example, 1331 is a palindrome.
    
    Disprove: all palidromes with two or more digits are
    divisible by $11$.
\end{question}
%<*answers>
%% Student: put your answer between the next two lines.
\begin{answer}
\end{answer}
%</answers>
%<*solutions>
\begin{solution}
    Let $n = 101$.  Then $n$ is a palindrome.
    But $9\times 11 = 99$ and $10\times 11 = 110$.  Since there are no multiples of $11$ between these, $101$ is not prime.
\end{solution}
%</solutions>


\begin{question}
    Scheinerman, Exercise 7.12:
    Another method to prove that certain Boolean formulas
    are tautologies is to use the properties in Theorem~7.2
    together with the fact that $x \rightarrow y$ is equivalent
    to $(\neg x) \vee y$ (Proposition~7.3)
    For example, Exercise 7.11, part (b) asks you to establish that the formula $(x \wedge (x \rightarrow y)) \rightarrow y$ is 
    a tautology.  Here is a derivation of that fact:
    % This is an "align" environment for aligning
    % mathematical equations.  Ampersands (&) denote 
    % separation between columns.  The first one means
    % there will be alignment around the equal sign.
    % The double-ampersands put a second column, left-
    % justified.  Double-backslashes (\\) separate lines.
    \begin{align*}
        (x \wedge (x \rightarrow y)) \rightarrow y
        &= [x \wedge (\neg x \vee y)] \rightarrow y
        && \text{translate $\rightarrow$} 
        \\
        &= [(x \wedge \neg x) \vee (x \wedge y)] \rightarrow y
        && \text{distributive}
        \\
        &= [\mathrm{FALSE} \vee (x\wedge y)] \rightarrow y
        && \text{inverse elements}
        \\
        &= (x\wedge y) \rightarrow y
        && \text{identity element}
        \\
        &= \neg(x\wedge y) \vee y
        && \text{translate $\rightarrow$}
        \\
        &= (\neg x \vee \neg y) \vee y
        && \text{De~Morgan's laws}
        \\
        &= \neg x \vee (\neg y \vee y)
        && \text{associativity}
        \\
        &= \neg x \vee \mathrm{TRUE}
        && \text{inverse elements}
        \\
        &= \mathrm{TRUE}
        && \text{identity element}
        \\
    \end{align*}
    Use this technique [not truth tables] to prove that these formulas
    are tautologies:
    \begin{enumerate}
        \item $(x \vee y) \vee (x \vee \neg y)$
        \item $(x \wedge (x \rightarrow y))\rightarrow y$
        \item $\neg(\neg x) \leftrightarrow x$
        \item $x \rightarrow x$
        \item $((x \rightarrow y) \wedge (y \rightarrow z)) )\rightarrow (x \rightarrow z)$
        \item $\mathrm{FALSE} \rightarrow x$
        \item $(x \rightarrow \mathrm{FALSE}) \rightarrow \neg x$
        \item $(x \rightarrow y) \wedge (x \rightarrow \neg y) \rightarrow \neg x$
    \end{enumerate}
\end{question}
%<*answers>
%% Student: put your answer between the next two lines.
\begin{proof}
\end{proof}
%</answers>
%<*solutions>
\begin{proof}
    \begin{enumerate}
        \item Factor out $x\vee$ from both:
        \[
            (x \vee y) \vee (x \vee \neg y)
            = x \vee (y \vee \neg y) = x \vee \mathrm{TRUE} = \mathrm{TRUE}
        \]
        \item Rewrite the $\to$ in terms of $\vee$:
        \begin{align*}
            (x \wedge (x \rightarrow y))\rightarrow y
            &= \neg(x \wedge (x \rightarrow y))\vee y 
            \\&= \neg x \vee \neg(\neg x\vee y)) \vee y 
            \\&= (\neg x \vee (x \wedge \neg y)) \vee y 
            \\&= ((\neg x \vee x) \wedge (\neg x \vee \neg y)) \vee y
            \\&= (\mathrm{TRUE} \wedge (\neg x \vee \neg y)) \vee y
               = (\neg x \vee \neg y) \vee y
            \\&= \neg x \vee (\neg y \vee y)
               = \neg x \vee \mathrm{TRUE}
               = \mathrm{TRUE}
        \end{align*}
        \item If we allow that $x \rightarrow x = \mathrm{TRUE}$ (the next part),
        then
        \[
            \neg (\neg x) \leftrightarrow x
            = x \leftrightarrow x
            = (x \rightarrow x) \wedge (x \rightarrow x)
            = \mathrm{TRUE} \wedge \mathrm{TRUE} = \mathrm{TRUE}
        \]
        
        \item Now
        \(
            x\rightarrow x = \neg x \vee x = \mathrm{TRUE}
        \)
        \item
        \begin{align*}
            ((x \rightarrow y) \wedge (y \rightarrow z)) )\rightarrow (x \rightarrow z)
            &= \neg((\neg x \vee y) \wedge (\neg y \vee z))\vee (\neg x \vee z)
          \\&= (\neg(\neg x \vee y) \vee \neg (\neg y \vee z))\vee (\neg x \vee z)
          \\&= (x \wedge \neg y) \vee (y \wedge \neg z) \vee (\neg x \vee z)
          \\&= (\neg x \vee (x \wedge \neg y)) 
               \vee ((y \wedge \neg z) \vee  z)
          \\&= ((\neg x \vee x) \wedge (\neg x \vee \neg y))
               \vee ((y \vee z) \wedge (\neg z \vee z))
          \\&= (\mathrm{TRUE} \wedge (\neg x \vee \neg y))
               \vee ((y \vee z) \wedge \mathrm{TRUE})
          \\&= (\neg x \vee \neg y) \vee (y \vee z)
          \\&= \neg x \vee (\neg y \vee y) \vee z
          \\&= \neg x \vee \mathrm{TRUE} \vee z = \mathrm{TRUE}
        \end{align*}
        \item $\mathrm{FALSE} \rightarrow x = \mathrm{TRUE} \vee x = \mathrm{TRUE}$
        \item
        \begin{align*}
            (x \rightarrow \mathrm{FALSE}) \rightarrow \neg x
            &=\neg(\neg x \vee \mathrm{FALSE}) \vee \neg x
          \\&=(x \wedge \mathrm{TRUE}) \vee \neg x 
          \\&=(x \vee \neg x)\wedge(\mathrm{TRUE} \vee \neg x) 
          \\&= \mathrm{TRUE} \wedge \mathrm{TRUE} = \mathrm{TRUE}
        \end{align*}
        \item
        \begin{align*}
            (x \rightarrow y) \wedge (x \rightarrow \neg y) \rightarrow \neg x
            &= \neg((\neg x \vee y) \wedge (\neg x \vee \neg y)) \vee \neg x
          \\&= (\neg(\neg x \vee y) \vee \neg(\neg x \vee \neg y)) \vee \neg x
          \\&= ((x \wedge \neg y) \vee (x \wedge y)) \vee \neg x
          \\&= (x \wedge (\neg y \vee y))\vee \neg x
          \\&= (x \wedge \mathrm{TRUE}) \vee \neg x
          \\&= x \vee \neg x = \mathrm{TRUE}
        \end{align*}
    \end{enumerate}
\end{proof}
%</solutions>

%<*skip>
\begin{question}
    Scheinerman, Exercise 7.19:
    Prove that $x \vee y$ can be expressed in terms of
    just $\wedge$ and $\neg$, so all boolean operations
    can be reduced to just two.
\end{question}
% Student: put your answer between the next two lines.
\begin{solution}
\end{solution}

\begin{question}
    Scheinerman, Exercise 7.20:
    Here is yet another boolean operation called 
    \emph{nand}.  It is denoted by the symbol $\bar\wedge$.
    We define $x \bar\wedge y$ to be $\neg(x\wedge y)$.
    Do the following:
    \begin{enumerate}
        \item Construct a truth table for $\bar\wedge$.
        \item Is the operation $\bar\wedge$ commutative? Associative?
        \item Show how the operations $x \rightarrow y$
        and $\neg x$ can be expressed just in terms of $\bar\wedge$.
        \item Conclude that all binary boolean operations 
        can be expressed just in terms of $\bar\wedge$ alone.
    \end{enumerate}
\end{question}
% Student: put your answer between the next two lines.
\begin{solution}
\end{solution}
%</skip>

\begin{question}
    Scheinerman, Exercise 8.13:
    Let $n$ be a positive integer.
    Prove that $n^2 = (n)_2 + n$ in two different ways.

    First (and more simply), show this equation is true algebraically.
    
    Second (and more interestingly), interpret the terms $n^2$, $(n)_2$, and $n$
    in the context of list counting as use that to argue why the equation must
    be true.
\end{question}
%<*answers>
%% Student: put your answer between the next two lines.
\begin{answer}
\end{answer}
%</answers>
%<*solutions>
\begin{solution}
    \begin{enumerate}
        \item The algebraic proof:
        \[
            (n)_2 + n = n(n-1) + n = n^2-n+n = n^2
        \]
        \item the combinatorial proof: Consider all lists of elements $(i,j)$, where $i$ and $j$ come from $1,2,\dots,n$.  There are obviously $n^2$ of these.
        But there are $n$ of them which satisfy $i=j$, and $(n)_2$ of them which satisfy $i \neq j$.  So $n^2 = (n)_2 + n$.
    \end{enumerate}
\end{solution}
%</solutions>

\begin{question}
    Scheinerman, Exercise 8.15:
    How many five-digit numbers are there that do not have two consecutive digits
    the same?  For example, you would count 12104 and 12397 but not 6321 (it is
    not five digits) or 43356 (it has two consecutive 3s).
    
    Note: the first digit may not be a zero.
\end{question}
%<*answers>
%% Student: put your answer between the next two lines.
\begin{answer}
\end{answer}
%</answers>
%<*solutions>
\begin{solution}
    The first digit has nine choices: anything from $1$ to $9$.
    The second digit has nine choices, too: anything from $0$ to $9$, but not counting whatever the first digit is.
    The same is true of the third, fourth, and fifth digits.
    Therefore, by the Multiplication Principle, the number of numbers is $9^5$.
\end{solution}
%</solutions>


\begin{question}
    Scheinerman, Exercise 8.19:
    Four cards are drawn from a standard deck of 52 cards.
    In how many ways can this be done if the cards are all of different values
    (e.g., no two 5s or two jacks) and all different suits?
    (For this problem, the order in which the cards are drawn matters,
    so drawing A$\spadesuit$-K$\heartsuit$-3$\diamondsuit$-6$\clubsuit$
    is not the same a drawing 6$\clubsuit$-K$\heartsuit$-3$\diamondsuit$-A$\spadesuit$
    even though the same cards are selected.)
\end{question}
%<*answers>
%% Student: put your answer between the next two lines.
\begin{answer}
\end{answer}
%</answers>
%<*solutions>
\begin{solution}
    There are 13 ranks and 4 suits.  
    The first card can be any rank and any suit.
    The second card may be of any of the 12 remaining ranks and 3 remaining suits.
    The third card may be of any of the 11 remaining ranks and 2 remaining suites.
    The fourth card may be of any of the 10 remaining ranks and 1 remaining suit.
    So the number of hands is
    \[
        (13)_4 \times 4! = \num{411840}
    \]
\end{solution}
%</solutions>


\begin{question}
    Scheinerman, Exercise 9.8:
    Calculate the following products.
    \begin{enumerate}
        \item $\prod_{k=1}^4(2k+1)$
        \item $\prod_{k=-3}^4k$
        \item $\prod_{k=1}^n \frac{k+1}{k}$, where $n$ is a positive integer
        \item $\prod_{k=1}^n \frac{1}{k}$, where $n$ is a positive integer
    \end{enumerate}
\end{question}
%<*answers>
%% Student: put your answer between the next two lines.
\begin{answer}
\end{answer}
%</answers>
%<*solutions>
\begin{solution}
    \begin{enumerate}
        \item \[\prod_{k=1}^4(2k+1)
            = (2\times 1 + 1)(2 \times 2 +1)(2 \times 3 +1(2\times 4+1)
            = 3 \times 5 \times 7 \times 9 = 945
        \]
        \item \[\prod_{k=-3}^4k
            =(-3)(-2)(-1)(0)(1)(2)(3)(4)=0
        \]
        \item When expanding the product, everything cancels except
        the last numerator and the first denominator:
        \[\prod_{k=1}^n \frac{k+1}{k} =
           \frac{2}{1}\times\frac{3}{2}\times\frac{4}{3}
           \times\dots\times\frac{n+1}{n} = n+1
        \]
        \item \[
            \prod_{k=1}^n \frac{1}{k}
            = \frac{1}{1} \times \frac{1}{2} \times \cdots \times \frac{1}{n} = \frac{1}{n!}
        \]
    \end{enumerate}
\end{solution}
%</solutions>


\begin{question}
    Scheinerman, Exercise 9.10:
    When $100!$ is written out in full, it equals
    \[
        100! = 9332621\dots000000
    \]
    Without using a computer, determine the number of $0$ digits at the end
    of this number.
\end{question}
%<*answers>
%% Student: put your answer between the next two lines.
\begin{answer}
\end{answer}
%</answers>
%<*solutions>
\begin{solution}
    In $1\times 2\times 3\times \cdots\times100$ there are $24$ factors of $5$:
    $20$ from the numbers $5$, $10$, $15$, $\ldots$, plus $4$ more from $25$,
    $50$, $75$, and $100$. There are more factors of $2$. So all told there are
    $24$ factors of $10$ in $100!$, so in base-10, $100!$ ends in $24$ zeros.
\end{solution}
%</solutions>

\end{document}
%</questions>